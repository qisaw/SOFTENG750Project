\documentclass[a4paper]{article}

\usepackage{url}
\usepackage{apacite}

\begin{document}

\title{Parallelization Potential of Existing Serial Code}
\author{Mikhail D'Souza, Wasiq Kashkari, Sara Metz}
% \date{12/03/2014}
\maketitle

\abstract{Now that multi-core mobile phones are becoming commonplace, there is
more emphasis on multi-core aware programming. This article proposes creating a
tool that allows a developer to see how much of his/her code will benefit from
parallelization. This presents many challenges such as being able to visualize the
benefits of parallelization meaningfully and finding and using good metrics to
gauge how much of the code can be parallelized. With the help of PDStore, this project will attempt to 
analyse parallelizability at a superficial level and output to the user,
graphs of the speed-up benefits of different methods.}

\section{Research Question}
\begin{description}
\item Is it feasible to predict the amount of code that can be parallelized?
\item How do we get a visual representation of the benefits of parallelizing a
piece of code?
\end{description}

\section{Motivation}
In terms of frequency, processors are beginning to peak at their practical
limits. The development and widespread public use of devices with
multiple cores is rapidly increasing. Whilst these multi-core processors are
capable of running multiple tasks asynchronously, the majority of code in the
industry is not designed to efficiently capitalize on this capability. These
physically superior devices are being used inefficiently.

Developers are beginning to adapt to the changing hardware landscape.
Sequential code is being adjusted for multi-threaded environments. This is
complicated and time consuming, especially if a developer lacks prior knowledge
of the workings of multiprocessing. As a result, developing for multi-threaded
environments can become difficult. Developers are then faced with an unclear
view of the cost vs benefits of parallelizing a block of code.

Our research aims to give developers a tool to quickly and intuitively
visualize the benefits of parallelizing a selected block of code. The point of this
exercise is not to find an accurate description of how much can be parallelized
but rather to find the methods within the selected section of code that could benefit the most
from parallelization. Methods with the most promise of speedup from parallelization
could be improved upon first while methods with very little speedup could be neglected.

\section{Requirements}
\begin{description}
\item The program must use an efficient algorithm to analyse code blocks
and determine whether or not that block of code is parallelizable.
\item The program must take dependencies into account at some level
\item Give the user a quick and easy way to obtain a useful visualization
of the possible percentage speedup of a block of code.
\item Meaningfully break down the code into fractions that can and cannot
be parallelized
\end{description}

\section{Related Articles}
There are tools in industry that try to do some automatic parallelization
of selected loops based on dependence analysis. This has been done using both static
analysis \cite{helixProject} and runtime dynamic analysis \cite{1207020} 
\cite{Rus05hybriddependence}. However, these tools simply try to
parallelize all the code that they can, without taking software metrics such as
understandability and extensibility into account. This can result in difficult
to maintain code, which costs the developer more in the long run.

Furthermore, a programmer has knowledge about their program that the algorithm cannot
obtain, such as data composition and likely flow paths. Armed with such
knowledge, programmers have the ability to make better judgments on the
usefulness and most appropriate method of parallelization. This statement is
echoed by other research that states that automatic parallelization does not do
a good job in many situations \cite{5470884}. Hence, we propose an approach that
statically analyses source code, giving the programmer ideas of where
parallelization could be beneficial. But ultimately, leaving parallelization to
the programmer who can draw on their personal knowledge of the program.

\section{Design}
The creation of this tool will be based around an agile development process
where features will be developed and tested in an incremental fashion. This
will simplify the creation of a complicated tool into a set of relatively simple
features. With such a complex tool and a relatively short timespan, we shall create a simple tool that focuses on a subset of features in the Java language. However, if time permits, we may add more complex algorithms and checks.

The long term scope of the project is to create a tool that outputs a meaningful analysis of the amount of parallelizable code and the resulting speedup for a section of code selected by the user. This will be achieved at first by importing the code into PDStore, parsing through the abstract syntax trees it creates and obtaining a
superficial count of how many atomic operations occur in a block of code. The
assumption is that atomic lines of code cannot be parallelized internally while
most others can. This allows the developer to get a rough estimate of the speedup
they can expect by parallelizing their code. As the project progresses we intend
to include other methods such as dependence analysis to improve the accuracy of
this estimate.

\section{Evaluation}
\begin{description}
\item Use simple parallelization problems that the fraction of parallelizable
code can easily be calculated for. These fractions will then be used to compare
with the output of our program.
\item Evaluate the correctness of the solution by running simple problems on a
multi-core machine both serially and after being parallelized to compare the
actual speedup with the calculated values.
\item Run code from previous assignments through the program. The quality and usefulness of the graphs produced
will be evaluated.
\item Evaluate the solution by carrying out a user study of the usefulness of
the produced visualizations in comparison to current parallelizable code
visualization tools.
\end{description}
 
\section{Project Plan}
\begin{description}
\item \textbf{Milestone 1} - 11/04/2014 \\*
Source code parsing and command line input handling complete\\*
Unit tests written \\*
Initial metrics designed\\*
Visualization design complete
\item \textbf{Milestone 2} - 25/05/2014\\*
Parallelization analysis using initial metrics implemented\\*
Design for improvement of metrics completed\\*
Some initial visualization output
\item \textbf{Milestone 3}  - 16/05/2014\\*
Implemented improved metrics for parallelization detection \\*
Visualization output complete
\item \textbf{Milestone 4} - 20/05/2014\\*
User test completed\\*
All tests pass
\end{description}

\bibliographystyle{apacite}
\bibliography{refs}
\end{document}