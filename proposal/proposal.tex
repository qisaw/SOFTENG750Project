\documentclass[a4paper]{article}

\usepackage{url}
\usepackage{apacite}

\begin{document}

\title{Parallelisation Potential of Existing Serial Code}
\author{Mikhail D'Souza, Wasiq Kashkari, Sara Metz}
% \date{12/03/2014}
\maketitle

\abstract{Now that multi-core mobile phones are becoming the norm, there is
more emphasis on multi-core aware programming. This article proposes creating a tool that allows 
a developer to see how much of his/her code can be parallelised. This presents many challenges such as being able to 
visualize the benefits of parallelisation meaningfully and finding and using good metrics to
gauge how much of the code can be parallelised. This project will attempt to analyse parallelisability at a superficial level and pave the way
for more advanced heuristics to come.}

\section{Research Question}
\begin{description}
\item Is it feasible to predict the amount of code that can be parallelized?
\item How do we get a visual representation of the benfits of parallelizing a
piece of code?
\end{description}

\section{Motivation}
In terms of frequency, processors are beginning to peek at their physical
limits. The development and widespread public use of devices with
multiple cores is increasing rapidly. Whilst these multi-core processors are
capable of running multiple tasks at asynchronously, the majority of code in the
industry is not designed to efficiently capitalize on this capability. Thus,
these physically superior devices are inefficiently used since a large chunk of
processing power is wasted.

Developers are beginning to adapt to the changing hardware landscape.
Sequential code is being adjusted for multi-threaded environments. However this
is complicated and time consuming, especially if a developer lacks prior and
detailed knowledge of the inner workings of multiprocessing. As a result,
developing for multi-threaded environments can become difficult. Developers are
then faced with an unclear view of the cost vs benefits of parallelising a block
of code.

Our research question aims to give developers a tool to quickly and intuitively
visualise the benefits of parallelising a block of code. The point of this
exercise is not to find an accurate description of how much can be parallelised
but rather to find the methods within a package that could benefit the most from
parallelisation. Methods with the most promise of speed up from parallelisation
can be worked upon first while method with very little speedup can be neglected.

\section{Requirements}
\begin{description}
\item The program must use an efficient, algorithm to analyse code blocks
and determine whether or not that block of code is parallelizable.
\item The program must take dependencies into account to some level
\item Give the user a quick and easy way to obtain a useful visualisation
of the possible percentage speedup of a block of code.
\item Meaningfully break down the code into fractions that can and cannot
be parallelised
\end{description}

\section{Related Articles}
There are some tools in industry that try and do some automatic parallelisation
of some loops based on dependency analysis. This has been done using both static
analysis \cite{helixProject} and runtime dynamic analysis \cite{peterson1993} \cite{1207020} \cite{Rus05hybriddependence}. However, these
tools simply try to parallelise all the code that they can, without taking
software metrics such as understandability and extensibility into account. This
can result in difficult to maintain code, which costs the developer more in the
long run.

Furthermore, a programmer has knowledge about their program the algorithm cannot obtain, such as
data compisition and likely flow paths. Armed with such information, programmers the ability to make better judgements on the usefulness and most appropriate method of parallelisation. This statement has been supported by the authors of such automatic
paralellization tools who state that automatic paralellisation does not do a good job in many situations \cite{5470884}. Hence, we propose an approach that statically analyses source code, giving the programmer ideas of where parallelisation could be beinifical. But ultimately, leaving parallelization to the programmer who can draw on his or her personal knowledge of the program.

\section{Design}
The creation of this tool will be based around an agile development process where features will be developed and tested incremental fashion. This will simplify the creation of a complicated tool into a set of relatively simple features. With such a complex tool and a relatively short timespan, an agile process allows us to change the scope of the project very quickly and still have a fully funcitonal set of capabilities.

The long term scope of the project is to create a tool that outputs a meaningful analysis the amount of parallelisable code and the resulting speedup given a java derectory as input. This will be achieved at first by obtaining a superficial count of how many atomic operations occur in a block of code. The assumptions is that atomic lines of code cannot be parallelised while most others can. This allows the developer to get a rough estimate of the speed up they can expect by parallelizing their code. As the project progresses we intend to include other methods such as dependence analysis to improve the accuracy of this estimate.

\section{Evaluation}
\begin{description}
\item The first method will use simple parallelization
problems that the fraction of parallelizable code can easily be calculated for.
These fractions will then be used to compare with the solution�s output. 
\item Second method to evaluate the correctness of the solution will be to run
the same simple problems on a multicore machine both serially and after being
parallelized and to compare the perceived speedup values with the calculated
speedup values produced by the solution.
\item Third method is to get code from preveious assignments and run it through
the program. The output of the program will be judged. The quality and usefulness of
the graphs produced will be evaluated.
\item The final method that will be used to evaluate the solution will be to do
a user study of the usefulness of the produced visualizations in comparison to current
parallelizable code visualization tools.
\end{description}
 
\bibliographystyle{apacite}
\bibliography{refs}
\end{document}