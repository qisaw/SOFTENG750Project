\documentclass[a4paper,12pt]{article}

\usepackage{url}

\begin{document}

\title{Parallelisation Potential of Existing Serial Code}
\author{Mikhail D'Souza, Wasiq Kashkari, Sara Metz}
% \date{12/03/2014}
\maketitle

\abstract{Now that multi-core mobile phones are becoming the norm, there is
more emphasis on multi-core aware programming. This article looks are answers
the question, ``how much of the code can be parallelised.'' Finding and using
good metrics to gauge how much of the code can be parallelised is very hard.
This project analyse parallelisability at a superficial level and pave the way
for more advanced heuristics to come}

\section{Research Question}
Is it feasible to predict the amount of code that can be parallelized?

\section{Motivation}
In terms of frequency, processors are beginning to peek at their physical
limits. As a result, the development and widespread public use of devices with
multiple cores is increasing rapidly. Whilst these multi-core processors are
capable of running multiple tasks at asynchronously, the majority of code in the
industry is not designed to efficiently capitalize on this capability. Thus,
these physically superior devices are inefficiently used since a large chunk of
processing power is wasted.

Naturally, developers are beginning to adapt to the changing hardware landscape.
Sequential code is being adjusted for multi-threaded environments. However this
is complicated and time consuming , especially if a developer lacks prior and
detailed knowledge of the inner workings of multiprocessing. As a result,
developing for multi-threaded environments can become unnecessarily difficult.
Developers are then faced with an unclear view of the cost vs benefits of
parallelising a block of code.

Our research question aims to give developers a tool to quickly and intuitively
visualise the benefits ofparallelising a block of code. As a result, developers
will tend to become more productive in the long run as they will be more
reliably able to focus on parallelising areas of code with the most to benefit
from it.

The point of this exercise is not to find an accurate description of how much
can be parallelised but rather to find the methods within a package that could
benefit the most from parallelisation. Methods with the most promise of speed up
from parallelisation can be worked upon first while method with very little
speedup can be neglected.

\section{Requirements}
\begin{description}
\item The program must use an efficient, algorithm to analyse code blocks
and determine whether or not that block of code is parallelizable.
\item The program must take dependencies into account to some level
\item Give the user a quick and easy way to obtain a useful visualisation
of the possible percentage speedup of a block of code.
\item Meaningfully break down the code into fractions that can and cannot
be parallelised
\item Given a java package, produce some form of speedup vs number of
processors graph
\end{description}

\section{Related Articles}
Bla

\section{Design}
We will start design on the project in an agile manner. The plan is to churn out
features in an incremental fashion while testing them before moving on to the
next. The grand plan is to create a program that will be inputed a directory
full of Java code. The output will be an analysis of how much of the code in
there can be parallelised and what will be the resulting speedup for each piece
of code.

The program will calculate the total atomic lines of code. This metric will give
a good superficial count of how many atomic operations occur in a given piece of
code. The assumptions is that atomic lines of code cannot be parallelised within.
The second assumption is that all atomic lines of code that unit time to
execute. The point of this exercise is to get a very rough idea of the best case
scenario of how much time a certain piece of code will take to execute.

\section{Evaluation}
Evaluation of the success of the solution will be achieved using several
methods. The first method will use simple parallelization problems that the
fraction of parallelizable code can easily be calculated for. These fractions
will then be used to compare with the solution�s output. Another method to
evaluate the correctness of the solution will be to run the same simple problems
on a multicore machine both serially and after being parallelized and to compare
the perceived speedup values with the calculated speedup values produced by the
solution. The final method that will be used to evaluate the solution will be to
do a user study of the usefulness of the produced visualizations in comparison
to current parallelizable code visualization tools.

\end{document}